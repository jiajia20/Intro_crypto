%%% File encoding is ISO-8859-1 (also known as Latin-1)
%%% You can use special characters just like �,� and �

% Input encoding is 'latin1' (Latin 1 - also known as ISO-8859-1)
% CTAN: http://www.ctan.org/pkg/inputenc
% 
% A newer package is available - you may look into:
% \usepackage[x-iso-8859-1]{inputenc}
% CTAN: http://www.ctan.org/pkg/inputenx
\usepackage[latin1]{inputenc}

% Font Encoding is 'T1' -- important for special characters such as Umlaute � or � and special characters like � (enje)
% CTAN: http://www.ctan.org/pkg/fontenc
\usepackage[T1]{fontenc}

% Language support for 'english' (alternative 'ngerman' or 'french' for example)
% CTAN: http://www.ctan.org/pkg/babel
\usepackage[english]{babel} 

% Doing calculations with LaTeX units -- needed for the vertical line in the footer
% CTAN: http://www.ctan.org/pkg/calc
\usepackage{calc}

% Extended graphics support 
% There is also a package named 'graphics' - watch out!
% CTAN: http://www.ctan.org/pkg/graphicx
\usepackage{graphicx}

% Extendes support for floating objects (tables, figures), adds the [H] placing option (\begin{figure}[H]) which palces it "Here" (without any doubt).
% CTAN: http://www.ctan.org/pkg/float
\usepackage{float}

% Extended color support
% I use the command \definecolor for example. 
% Option 'Table': Load the colortbl package, in order to use the tools for coloring rows, columns, and cells within tables.
% CTAN: http://www.ctan.org/pkg/xcolor
\usepackage[table]{xcolor} 

% Nice tables
% CTAN: http://www.ctan.org/pkg/booktabs
\usepackage{booktabs}

% Better support for ragged left and right. Provides the commands \RaggedRight and \RaggedLeft. 
% Standard LaTeX commands are \raggedright and \raggedleft
% http://www.ctan.org/pkg/ragged2e
\usepackage{ragged2e}

% Create function plots directly in LaTeX
% CTAN: http://www.ctan.org/pkg/pgfplots
\usepackage{pgfplots}
\pgfplotsset{compat=1.11}

% Math stuff.
\usepackage{amsmath,amsfonts,amssymb}

% Code stuff.
\usepackage{listings}

\definecolor{mygreen}{rgb}{0,0.6,0}
\definecolor{mylightgray}{gray}{0.98}
\definecolor{mygray}{rgb}{0.5,0.5,0.5}
\definecolor{mymauve}{rgb}{0.58,0,0.82}
\lstset{ %
  backgroundcolor=\color{mylightgray},   % choose the background color; you must add \usepackage{color} or \usepackage{xcolor}; should come as last argument
  basicstyle=\footnotesize\ttfamily,        % the size of the fonts that are used for the code
  breakatwhitespace=false,         % sets if automatic breaks should only happen at whitespace
  breaklines=true,                 % sets automatic line breaking
  captionpos=t,                    % sets the caption-position to top
  commentstyle=\color{mygreen},    % comment style
  escapeinside={\%*}{*)},          % if you want to add LaTeX within your code
  extendedchars=true,              % lets you use non-ASCII characters; for 8-bits encodings only, does not work with UTF-8
  frame=single,	                   % adds a frame around the code
  keepspaces=true,                 % keeps spaces in text, useful for keeping indentation of code (possibly needs columns=flexible)
  keywordstyle=\color{blue},       % keyword style
  language=Octave,                 % the language of the code
  numbers=left,                    % where to put the line-numbers; possible values are (none, left, right)
  numbersep=5pt,                   % how far the line-numbers are from the code
  numberstyle=\tiny\color{mygray}, % the style that is used for the line-numbers
  rulecolor=\color{black},         % if not set, the frame-color may be changed on line-breaks within not-black text (e.g. comments (green here))
  showspaces=false,                % show spaces everywhere adding particular underscores; it overrides 'showstringspaces'
  showstringspaces=false,          % underline spaces within strings only
  showtabs=false,                  % show tabs within strings adding particular underscores
  stepnumber=2,                    % the step between two line-numbers. If it's 1, each line will be numbered
  stringstyle=\color{mymauve},     % string literal style
  tabsize=2,	                   % sets default tabsize to 2 spaces
  title=\lstname                   % show the filename of files included with \lstinputlisting; also try caption instead of title
}

% Stuff for the title page.
\usepackage{eso-pic}
\usepackage{transparent}